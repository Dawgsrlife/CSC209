\documentclass[12pt]{article}

\usepackage[utf8]{inputenc}
\usepackage{newunicodechar}
\usepackage{listings}
\usepackage{xcolor}
\usepackage{amsmath}
\usepackage{geometry}
\usepackage{setspace}

\geometry{letterpaper, portrait, includeheadfoot=true, hmargin=1in, vmargin=1in}
\onehalfspacing

% Configure listings for C code
\lstset{
    language=C,
    basicstyle=\ttfamily\small,
    keywordstyle=\color{blue}\bfseries,
    commentstyle=\color{green}\itshape,
    stringstyle=\color{orange},
    numberstyle=\color{gray},
    frame=single,
    breaklines=true,
    showstringspaces=false,
    numbers=left,
    numbersep=5pt,
    rulecolor=\color{gray},
}

\begin{document}

\section*{CSC209H Worksheet: Stacks and Heaps}

\subsection*{1. Trace the memory usage for the program below}

\textbf{Code:}
\begin{lstlisting}
#include <stdlib.h>
#include <stdio.h>

int *mkarray(int a, int b, int c) {
    int arr[3];
    arr[0] = a;
    arr[1] = b;
    arr[2] = c;
    int *p = arr;
    return p;
}

// Code for other_function() omitted.
int main() {
    int *ptr = mkarray(10, 20, 30);
    other_function();
    printf("%d %d %d\n", ptr[0], ptr[1], ptr[2]);
}
\end{lstlisting}

\textbf{Memory Trace:}
\begin{verbatim}
Section        Address    Value      Label
Heap           0x23c     ...        ...
               0x240     ...        ...
               0x244     ...        ...
Stack frame
for mkarray    0x454     ...        arr[0]
               0x458     ...        arr[1]
               0x45c     ...        arr[2]
               0x460     ...        p
               ...       ...        ...
Stack frame
for main       0x480     ...        ptr
               ...       ...        ...
\end{verbatim}

\subsection*{2. Why doesn't the program work?}
The program does not work because the array \texttt{arr} is allocated on the stack in \texttt{mkarray}, and the memory is invalidated after the function returns. This causes \texttt{ptr} in \texttt{main} to point to invalid memory.

\textbf{Fixed \texttt{mkarray} Function:}
\begin{lstlisting}
int *mkarray(int a, int b, int c) {
    int *arr = malloc(3 * sizeof(int));
    arr[0] = a;
    arr[1] = b;
    arr[2] = c;
    return arr;
}
\end{lstlisting}

\textbf{Updated Memory Trace:}
\begin{verbatim}
Section        Address    Value      Label
Heap           0x23c     10         arr[0]
               0x240     20         arr[1]
               0x244     30         arr[2]
Stack frame
for mkarray    0x454     ...        arr (pointer)
               ...       ...        ...
Stack frame
for main       0x480     0x23c      ptr
               ...       ...        ...
\end{verbatim}

\subsection*{3. Deallocate the memory}
The memory allocated on the heap should be deallocated to avoid memory leaks.

\textbf{Updated \texttt{main} Function:}
\begin{lstlisting}
int main() {
    int *ptr = mkarray(10, 20, 30);
    other_function();
    printf("%d %d %d\n", ptr[0], ptr[1], ptr[2]);
    free(ptr); // Deallocate memory
}
\end{lstlisting}

\subsection*{4. Trace the memory usage for the new program}

\textbf{Code:}
\begin{lstlisting}
#include <stdio.h>
#include <stdlib.h>

int *multiples(int x) {
    int *a = malloc(sizeof(int) * x);
    for (int i = 0; i < x; i++) {
        a[i] = (i + 1) * x;
    }
    return a;
}

int main() {
    int *ptr;
    int size = 3;
    ptr = multiples(size);
    for (int i = 0; i < size; i++) {
        printf("%d\t", ptr[i]);
    }
    printf("\n");
    return 0;
}
\end{lstlisting}

\textbf{Memory Trace:}
\begin{verbatim}
Section        Address    Value      Label
Heap           0x224     3          a[0]
               0x228     6          a[1]
               0x22c     9          a[2]
Stack frame
for multiples  0x46c     0x224      a (pointer)
               ...       ...        ...
Stack frame
for main       0x47c     0x224      ptr
               0x480     3          size
               ...       ...        ...
\end{verbatim}

\subsection*{5. Update \texttt{main} to handle multiple sizes and trace memory}

\textbf{Updated \texttt{main}:}
\begin{lstlisting}
int main() {
    int *ptr;
    for (int size = 3; size <= 5; size++) {
        ptr = multiples(size);
        for (int i = 0; i < size; i++) {
            printf("%d\t", ptr[i]);
        }
        printf("\n");
        free(ptr); // Deallocate memory
    }
    return 0;
}
\end{lstlisting}

\textbf{Memory Trace:}
\begin{verbatim}
For size = 3:
Heap           0x224     3          a[0]
               0x228     6          a[1]
               0x22c     9          a[2]
For size = 4:
Heap           0x230     4          a[0]
               0x234     8          a[1]
               0x238     12         a[2]
               0x23c     16         a[3]
For size = 5:
Heap           0x240     5          a[0]
               0x244     10         a[1]
               0x248     15         a[2]
               0x24c     20         a[3]
               0x250     25         a[4]
\end{verbatim}

\subsection*{6. Explanation of the problem}
Without the \texttt{free(ptr)} call, each iteration of the loop allocates memory on the heap without releasing it, leading to a memory leak. Adding \texttt{free(ptr)} after each use resolves the issue.

\end{document}
