\documentclass[12pt]{article}

\usepackage[utf8]{inputenc}
\usepackage{newunicodechar}
\usepackage{EngReport}
\usepackage{listings}
\usepackage{cancel}
\usepackage{comment}
\usepackage{amssymb}
\usepackage{amsthm}
\usepackage{amsmath}
\usepackage{graphicx}
\usepackage{setspace}
\usepackage{geometry}
\usepackage{xcolor}  % Required for coloring in listings

\graphicspath{{Images/}}
\onehalfspacing
\geometry{letterpaper, portrait, includeheadfoot=true, hmargin=1in, vmargin=1in}

% Define custom colors
\definecolor{myblue}{RGB}{0, 128, 255}
\definecolor{mygreen}{RGB}{34, 139, 34}
\definecolor{myorange}{RGB}{255, 140, 0}
\definecolor{mygray}{RGB}{128, 128, 128}
\definecolor{mypurple}{RGB}{148, 0, 211}
\definecolor{myred}{RGB}{255, 69, 0}

% Configure listings for Python with custom styles
\lstset{
    language=Python,             % Set language to Python
    basicstyle=\ttfamily\small,  % Use a smaller monospace font
    keywordstyle=\color{myblue}\bfseries,  % Keywords in blue and bold
    commentstyle=\color{mygreen}\itshape,  % Comments in green and italic
    stringstyle=\color{myorange},          % Strings in orange
    numberstyle=\color{mygray},            % Line numbers in gray
    identifierstyle=\color{mypurple},      % Functions and variables in purple
    morekeywords={print, len, range},      % Define additional Python keywords
    showstringspaces=false,                % Do not show spaces in strings
    breaklines=true,                       % Enable line breaking
    numbers=left,                          % Add line numbers to the left
    numbersep=5pt,                         % Space between line numbers and code
    frame=single,                          % Add a box around the code
    rulecolor=\color{mygray},              % Frame color
    moredelim=[is][\color{myred}]{@@}{@@}, % Custom inline LaTeX coloring
}

\begin{document}
\renewcommand{\familydefault}{\rmdefault}

\section*{CSC209H Worksheet: malloc Basics}

\subsection*{Question 1: Memory Allocation Table}
Each time a variable is declared or memory is allocated, it is important to understand how much memory is allocated, where it is allocated, and when it is deallocated. The table below summarizes this information:

\begin{table}[h!]
    \centering
    \begin{tabular}{|l|l|l|l|}
        \hline
        \textbf{Code Fragment} & \textbf{Space (Size)} & \textbf{Where Allocated} & \textbf{Deallocated When} \\ \hline
        \verb|int i;| & \verb|sizeof(int)| & Stack (main) & Program ends \\ \hline
        \verb|float i;| (in \verb|fun|) & \verb|sizeof(float)| & Stack (fun) & When \verb|fun| returns \\ \hline
        \verb|char i;| (in \verb|fun|) & \verb|sizeof(char)| & Stack (fun) & When \verb|fun| returns \\ \hline
        \verb|char i[10] = {'h', 'i'};| & \verb|10 * sizeof(char)| & Stack (main) & Program ends \\ \hline
        \verb|char *i;| & \verb|sizeof(char*)| & Stack (main) & Program ends \\ \hline
        \verb|int *i;| & \verb|sizeof(int*)| & Stack (main) & Program ends \\ \hline
        \verb|int i[5] = {4, 5, 2, 5, 1};| & \verb|5 * sizeof(int)| & Stack (main) & Program ends \\ \hline
        \verb|int *i = malloc(sizeof(int));| & \verb|sizeof(int)| & Heap & When \verb|free(i)| is called \\ \hline
        \verb|*i = malloc(sizeof(int) * 7);| & \verb|7 * sizeof(int)| & Heap & When \verb|free(i)| is called \\ \hline
    \end{tabular}
    \caption{Memory Allocation Table}
    \label{tab:memory_allocation}
\end{table}

\pagebreak

\subsection*{Question 2: Memory Trace}
Trace the memory usage for the program below up to the point when \verb|initialize| is about to return.

\begin{lstlisting}[language=C]
#include <stdio.h>
#include <stdlib.h>

void initialize(int *a1, int *a2, int n) {
    for (int i = 0; i < n; i++) {
        a1[i] = i;
        a2[i] = i;
    }
}

int main() {
    int numbers1[3];
    int *numbers2 = malloc(sizeof(int) * 3);
    initialize(numbers1, numbers2, 3);
    for (int i = 0; i < 3; i++) {
        printf("%d %d\n", numbers1[i], numbers2[i]);
    }
    free(numbers2);
    return 0;
}
\end{lstlisting}

\subsubsection*{Memory Layout}
Below is the memory layout at the point where \verb|initialize| is about to return:

\begin{table}[h!]
    \centering
    \begin{tabular}{|l|l|l|l|}
        \hline
        \textbf{Section}         & \textbf{Address} & \textbf{Value}             & \textbf{Label}          \\ \hline
        \textbf{Heap}            & 0x23c            & \verb|0|                   & \verb|numbers2[0]|      \\ \hline
                                 & 0x240            & \verb|1|                   & \verb|numbers2[1]|      \\ \hline
                                 & 0x244            & \verb|2|                   & \verb|numbers2[2]|      \\ \hline
        \textbf{Stack (initialize)} & 0x454         & Address of \verb|numbers1| & \verb|a1|               \\ \hline
                                 & 0x458            & Address of \verb|numbers2| & \verb|a2|               \\ \hline
                                 & 0x45c            & \verb|3|                   & \verb|n|                \\ \hline
                                 & 0x460            & \verb|0|                   & \verb|i| (loop counter) \\ \hline
        \textbf{Stack (main)}    & 0x474            & \verb|{0, 1, 2}|           & \verb|numbers1|         \\ \hline
                                 & 0x480            & \verb|0x23c| (Heap addr)   & \verb|numbers2|         \\ \hline
    \end{tabular}
    \caption{Memory Layout}
    \label{tab:memory_layout}
\end{table}

\end{document}
